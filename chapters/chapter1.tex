\makectitle

\chapter{\LaTeX{} 之我见}

\section{工具与艺术}

之前在人人上看到一篇介绍 \LaTeX{} 的文章,
第一句话说的便是
\begin{quotation}
  \it \LaTeX{} 并不难, \LaTeX{} 也不是艺术。
\end{quotation}
对此,我并不完全赞同。
诚然,对于一部分,甚至是很大一部分人来说
\LaTeX{} 的确只是一个写文章的工具,
但其底层 \TeX{} 设计的初衷绝非仅限于此。

\TeX{} 之父,正是有编程圣经之称的 {\it The Art of Computer Programming\/} 的作者,
大名鼎鼎的 \textit{Donald E. Knuth\/} 教授。
他在为 \TeX{} 系统撰写的手册 \textit{The TeXBook\/} 第一章便写道

\begin{quotation}
  \small
  English words like `technology' stem from a Greek root beginning with
  the letters $\tau\epsilon\chi\ldots\,$; and this same Greek word means {\sl
  art\/} as well as technology. Hence the name \TeX, which is an
  uppercase form of $\tau\epsilon\chi$.
  
  \begin{center}
      \ldots\ldots\ldots\ldots\qquad
      \ldots\ldots\ldots\ldots\qquad
      \ldots\ldots\ldots\ldots\qquad
      \ldots\ldots\ldots\ldots\qquad
      \ldots\ldots\ldots\ldots
  \end{center}
  
  The purpose of this pronunciation exercise is to remind you that \TeX\ is
  primarily concerned with high-quality technical manuscripts: Its emphasis is
  on art and technology, as in the underlying Greek word. If you merely want
  to produce a passably good document---something acceptable and basically
  readable but not really beautiful---a simpler system will usually suffice.
  With \TeX\ the goal is to produce the {\sl finest\/} quality; this requires
  more attention to detail, but you will not find it much harder to go the
  extra distance, and you'll be able to take special pride in the finished
  product.
\end{quotation}
足见 \TeX{}/\LaTeX{} 并不仅仅是工具而已,它的诞生就是为了满足高质量排版的各种苛刻要求。

又或者说,\LaTeX{} 本身是一种工具,而排版是艺术。
就好比一支笔。很多人用笔写字,只是为了记录和表达,于他们而言,笔仅仅是工具。
但也有人,成了书法家,书法是一种艺术。
要写得好字,功夫自是必不可少的,一支好笔也同样很重要。
\LaTeX{} 对于排版艺术,就是一支好笔。


\section{软件与语言}

就好比 C++ 是一种编程语言,而微软的 VC++ 是开发环境,它只是为了提高效率而已。
没有 VC++,还可以选择 devC++,或者使用任何文本编辑器配合编译器。
语言的核心,只是代码和编译器而已。

同样的,虽然很多人把 \LaTeX{} 视为软件,私以为以排版语言理解之更为合适。
其实只需一个编辑器编辑源文件,在命令行里编译,就可以得到高质量的排版结果。
而所说的软件,如 \CTeX 套装里自带的 WinEdt,以及开源的 TeXworks,TeXstudio 等,
就好比 VC++ 或者 devC++。
所以说,\LaTeX{} 其实是一种编译器,或者语言。
正如 C++ 编译器用来编译 C++ 语言,当谈论 \LaTeX{},我们既指编译器,也指语言。

当然,令人惊奇的是,当我们谈论 \LaTeX{},很多时候我们用的却并非在用 \LaTeX{} 进行编译。
首先,我们应该先搞清楚 TeX,pdfTeX,XeTeX 以及 LaTeX,pdfLaTeX,XeLaTeX。
当然,继续了解,还有 ConTeXt,LuaTeX 等。
比如 TeX,pdfTeX,XeTeX,是 TeX 的三种编译器,它们共同编译满足 TeX 语法的源文件。
不同的是 TeX 输出的是~.dvi 格式的文件,需要经过额外的转换才能得到 pdf 文件。
要知道,TeX 的诞生可比 pdf 的流行要早得多。
而 pdfTeX,顾名思义,可以直接输出 pdf 文件。
TeX 系统原本并不支持调用操作系统字体,它调用的字体都是 TeX 系统自带的文件定义的,
而标准的 LaTeX 发行版,因为版权原因并没有包含 windows 上常用中易字体的定义文件。
另一方面,原先的 TeX 系统只具备处理标准 ASCII 码的能力,
对于中文这样字符数量庞大的语言,就更加无法提供支持了。
虽然 TeX 系统诞生之出并不支持非英文语言,
但在世界各地用户的努力下,TeX 系统也开始支持越来越多的语言。
至于 XeTeX,则可以调用系统字体,并且使用通用性更好的 UTF-8 编码,让非英文排版变得简单。
再说说我们的主角 LaTeX,原本只是基于 TeX 的一个拓展。
它的语法与 TeX 并不兼容。虽然 TeX 文档和 LaTeX 文档的后缀都是~.tex,
但既不能用 TeX 编译器编译 LaTeX 源文件,也不能用 LaTeX 编译器编译 TeX 源文件。
LaTeX 和 ConTeXt,LuaTeX,都是 TeX 的拓展。
LaTeX 是其中最流行的拓展,另外两个不甚了解,就不作介绍了。
这里已经不难理解 pdfLaTeX 和 XeLaTeX,分别是 pdfTeX 和 XeTeX 向上兼容 LaTeX 拓展的产物了。

当然 \LaTeX{} 现在已经是一个很笼统的称呼了。
一个完整的 \LaTeX{} 发行套装,包含上述提到的全部编译器,
甚至还提供了 \texttt{metapost} 和 Asymptote 这样强大的绘图语言。
另外 windows 版本还会自带编辑器。
而我们最常用的还是 \LaTeX{} 语言和它相应编译器。
由于以前用 Linux 时 LaTeX 和 pdfLaTeX 还不能处理中文,
我自己对 XeLaTeX 情有独钟,
以至于这份模板仅支持 XeLaTeX 一种编译途径。


\section{\LaTeX{} 与 Word}

\LaTeX{} 与 Word 的差别,
其实就是一个开源的可扩展的排版语言,
和一个封闭的商业软件的差别。

首先,作为一种排版语言,\LaTeX{} 体现出了惊人的可扩展性。
只要有人为它开发宏包,它就能拥有很多令人吃惊的功能。
除了自带的优秀的公式排版,它还可以借助相应的宏包排版
化学结构式、流程图、交换图,甚至各种棋谱和乐谱。
使用 Word,你需要另外购买的功能,\LaTeX{} 很多都可以通过宏包实现拓展。

同时,作为排版语言,\LaTeX{} 让你可以通过命令对排版效果进行细致的控制。
就比如上边这段话,我们可以试着在“甚至各种棋谱和乐谱。”
后加上 \verb|\linebreak| 命令来建议 \LaTeX{} 断行。
得到的效果如下边这段话:

首先,作为一种排版语言,\LaTeX{} 体现出了惊人的可扩展性。
只要有人为它开发宏包,它就能拥有很多令人吃惊的功能。
除了自带的优秀的公式排版,它还可以借助相应的宏包排版
化学结构式、流程图、交换图,甚至各种棋谱和乐谱。\linebreak
使用 Word,你需要另外购买的功能,\LaTeX{} 很多都可以通过宏包实现拓展。

在 Word 里没有这样的自由吧,是不是很赞!再看看这个,如果我要输入公式
\begin{gather*}
  -x<y<z \\ p<y<z
\end{gather*}
这样的效果似乎还不够,在 $p$ 前加入 \verb|\phantom{-}| 占据一个负号的位置呢?
\begin{gather*}
  -x<y<z \\ \phantom{-}p<y<z
\end{gather*}
我想这就是工具与艺术的区别!

\LaTeX{} 另外个巨大的巨大的优势就在于用
\verb|\chapter{}|,\verb|\section{}| 这样的命令来整理文章的逻辑结构,
通过预先定义章节标题的样式来实现各种效果。
试想一下 Word 里,你打了“1. 标题”,然后回车准备输入正文,这时候出现了自动编号。
可是下一节写什么你还没有想好啊,于是退格删除,把格式改回正文的格式。
打到下一节,输入“2. 标题”,再用格式刷把第一节的标题刷到第二节。
为了让标题与上一段分开一点,你在上一段末尾多敲了一个回车。
有时候还会遇到更尴尬的情况,这个标题正好在一页的末尾,这一节就这样被“斩首”了,
不太好看,于是你回到上一节,啪啪啪一路回车愣是把这一节的标题换到下一页。
接着写,你突然想起来前面要再插入一节。
如果你不太清楚大纲视图和自动编号,那惨了,你得把后面的节前面的数字全部改了。
别忘了还有你之前换页敲的一溜回车,重新删了吧……

好吧,我又在意淫 Word 用户的困扰了。
其实我并不是要黑 Word,只是拿全靠回车、空格和格式刷排版的 Word 用户举个例子。
习惯了 \LaTeX{} 的 \verb|\chapter{}|,\verb|\section{}|这一套,
回到 Word,便会自觉地开始学习用样式来规定标题的格式和编号,
标题设置成与下段同页就避免了上述的尴尬,
孤行控制还可以防止某段的最后一行落到下一页,诸如此类。
当习惯了 \LaTeX{} 通过 \verb|\label{}| 和 \verb|\ref{}| 统一地实现交叉引用,
回到 Word,就会学习使用题注来添加图表标题。
当然,你最终会发现,这些没有 \LaTeX{} 里的命令来得好用。

这便是 \LaTeX{} 所谓“所想既所得”与 Word 之“所见即所得”的差别。
\LaTeX{} 让你专注于写文章,而非文章的排版。
我想很多人之所以把 \LaTeX{} 视为工具,原因就在于此。
既然不需要写文章的人关注排版,那么事先定义好详细的排版规则,
让排版出来的文章美观、高质量,这就是 \LaTeX{} 和其它宏包设计者该做的事了。
对于这些人,\LaTeX{} 必须是艺术。
当然,前面给出的两个关于细节控制的例子,
对于将 \LaTeX{} 完全视为工具的使用者而言,是不必要了解的,
对于重视排版效果的人,则会为这个系统贴心的设计感到欣喜。

写了那么多,并不是为了显示 \LaTeX{} 有多么优越。
Word 也很强大,只是它太容易上手了,
容易到用户会很容易无视它强大的功能而选择最“暴力”的办法。
当然,说到论文模板这种东西,真不得不说还是 \LaTeX{} 好用。
至少每次在 Word 的某条下划线上打字,字的位置还得尽量居中的时候,我都感到十分头大。
况且 Word 整理各种编号和文档结构的效率,还真是远不如 \LaTeX{}。
有人说 \LaTeX{} 是理科生写论文用的,甚至理科生里搞实验的会说那是搞理论的人用的。
我看其实不然,诚然 \LaTeX{} 高效的公式编辑和美观的公式排版确实对搞理论的人很有用,
但如果一个的文科生,一个脱离了公式和图表的纯粹的文科生,
用 \LaTeX{} 应该感到同样的舒心痛快,乃至更甚。
试想一下,一篇论文,只需要各级标题和引用,全文几乎只用到
\verb|\chapter{}| 和 \verb|\section{}| 开启新的章节,
\verb|\cite{}| 引用参考文献,\verb|\footnote{}| 添加脚注,
岂不是比用 Word 排版还要容易得多!
